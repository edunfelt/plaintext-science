\usepackage[utf8]{inputenc}

%================%
% Misc. packages %
%================%
\usepackage{amsmath, amssymb, amsfonts, amsthm}
\usepackage{graphicx}
\usepackage{hyperref}
\usepackage[ruled,vlined]{algorithm2e}  % pseudocode

%=================%
% Header & footer %
%=================%
\usepackage{fancyhdr}
\usepackage{sectsty}
\pagestyle{fancy}
\setlength{\headheight}{15pt}
\lhead{Emilia Dunfelt}\chead{CODE Assignment 1}\rhead{\today}
\lfoot{}\cfoot{\thepage}\rfoot{}

%=============%
% Eunmeration %
%=============%
\usepackage{enumitem}
\setlist{itemsep=0pt}
\renewcommand{\theenumi}{(\sl \alph{enumi})}
\renewcommand{\labelenumi}{\theenumi}
\renewcommand{\theenumii}{\roman{enumii}}

%=======================================%
% Counters for problems and subsections %
%=======================================%
\newcounter{ProblemNum}
\newcounter{SubProblemNum}[section]

\renewcommand{\theProblemNum}{\arabic{ProblemNum}}
\renewcommand{\theSubProblemNum}{\alph{SubProblemNum}}

\renewcommand*{\section}[1]{\stepcounter{ProblemNum} %
   {\newpage\sc Problem \theProblemNum. \; #1}}
\newcommand*{\soln}[1]{\paragraph*{#1}}
\renewcommand*{\subsection}[1]{\setcounter{SubProblemNum}{0} \soln{\sc Solution}}
\renewcommand*{\subsubsection}[1] %
    {\stepcounter{SubProblemNum} \soln{\sl Part (\theSubProblemNum)\newline}}

%======%
% Font %
%======%
\usepackage{libertine}
\usepackage[libertine]{newtxmath}

%========%
% Colors %
%========%
\newcommand{\TODO}[1]{\begingroup\color{Maroon}#1\endgroup}
\newcommand{\OLD}[1]{\begingroup\color{Gray}#1\endgroup}

%=====================%
% Theorem styles etc. %
%=====================%
\theoremstyle{plain}
\newtheorem{theorem}{Theorem}
\newtheorem{corollary}{Corollary}
\newtheorem{lemma}[theorem]{Lemma}
\theoremstyle{definition}
\newtheorem{definition}[theorem]{Definition}
\newtheorem{example}[theorem]{Example}
\newtheorem{remark}{Remark}
\newtheorem{claim}{Claim}
\newcommand\exend{{\hfill$\bigstar$}}
\newcommand\prfend{{\hfill$\blacksquare$}}

%=============%
% Common sets %
%=============%
\newcommand{\Z}{\mathbb{Z}}
\newcommand{\N}{\mathop{}\mathbb{N}}
\newcommand{\Q}{\mathop{}\mathbb{Q}}
\newcommand{\R}{\mathop{}\mathbb{R}}
\newcommand{\C}{\mathop{}\mathbb{C}}
